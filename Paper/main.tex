\documentclass[a4paper, UKenglish, cleveref, autoref, thm-restate]{lipics-v2021}

\bibliographystyle{plainurl}

\title{Fast Random Integer Generation in an Interval} 

\author{Lukas Geis}{Goethe Universität Frankfurt am Main}{s2949316@stud.uni-frankfurt.de}{}{}

\authorrunning{L. Geis} 

\Copyright{Lukas Geis}

\ccsdesc[100]{Mathematics of computing $\rightarrow$ Random number generation}

\keywords{Random Number Generation, Rejection Method, Randomized Algorithms, Algorithm Engineering} 

\category{} 
\relatedversion{} 
\supplement{Sources of implementations and experiments can be accessed at \url{https://github.com/lukasgeis/FRIGiaI-Review}}
\acknowledgements{}
\nolinenumbers 


%Editor-only macros:: begin (do not touch as author)%%%%%%%%%%%%%%%%%%%%%%%%%%%%%%%%%%
\EventEditors{John Q. Open and Joan R. Access}
\EventNoEds{2}
\EventLongTitle{42nd Conference on Very Important Topics (CVIT 2016)}
\EventShortTitle{CVIT 2016}
\EventAcronym{CVIT}
\EventYear{2016}
\EventDate{December 24--27, 2016}
\EventLocation{Little Whinging, United Kingdom}
\EventLogo{}
\SeriesVolume{42}
\ArticleNo{23}
%%%%%%%%%%%%%%%%%%%%%%%%%%%%%%%%%%%%%%%%%%%%%%%%%%%%%%

% Packages
\usepackage{mathtools}
\usepackage{nicefrac}
\newcommand{\Mod}[1]{\ \mathrm{mod}\ #1}

\usepackage[vlined,ruled,linesnumbered]{algorithm2e}
\DontPrintSemicolon
\SetKwInOut{Require}{Require}

\usepackage{hyperref}



\begin{document}

\maketitle


%%%%% Abstract
\begin{abstract}
    Lorem ipsum dolor sit amet, consectetur adipiscing elit. Praesent convallis orci arcu, eu mollis dolor. Aliquam eleifend suscipit lacinia. Maecenas quam mi, porta ut lacinia sed, convallis ac dui. Lorem ipsum dolor sit amet, consectetur adipiscing elit. Suspendisse potenti. 
\end{abstract}


%%%%% Introduction
\section{Introduction}
\subsection{Motivation}
The study of randomized algorithms goes as far back as the 1910s~\cite{MathComp}.
While initially small, this study is now scientifically pursued in almost all parts of computer science.
Not only in theory, but also in practice, randomized algorithms gain even more importance today to not only solve specific algorithmic problems but also to create large, scalable, and unbiased data sets that can be used in a variety of fields.
\\
The core of such algorithms is most often the pseudo-random number generator (PRNG).
There are many efficient PRNGs such as XorShift~\cite{XS}, MersenneTwister~\cite{MT}, linear congruential generators~\cite{LinConGen, MINSTD, LM}, and younger generators such as the PCG-Family~\cite{PCG} and ChaCha~\cite{Cha}.
These generators produce $32$-bit or $64$-bit words which we can interpret as unsigned integers in $[0,2^{32})$ and $[0,2^{64})$ respectively.
However, for most applications we do not want to have such a large range of possible random values but instead want to confine the range to a given interval $[a,b]$.
This can not only be used to uniformly sample an element from an array but also for more complex algorithms
\begin{itemize}
    \item The Fisher-Yates random shuffle~\cite{FY} (see \hyperref[alg:fy]{Algorithm 1}) uniformly permutes an array of $n$ elements in time $\Theta(n)$. 
    We have to draw a random integer in an interval for every shuffled element.
    \item A random graph model describes a distribution over a (parametrized) family of graphs. 
    There exist an uncountable number of such models and almost all of them require some sort of integer sampling in an interval. 
    For example, the $\mathcal{G}(n,m)$ model~\cite{GNM} uniformly samples a (directed) graph with $n$ nodes and $m$ edges. 
    A possible algorithm for this problem is Reservoir-Sampling~\cite{Reservoir}(see \hyperref[alg:gnm]{Algorithm 2}) which is a general technique that can sample $k$ elements without replacement from an input stream. 
    Another model is the BA model~\cite{BABook,BA} which iteratively builds the graph by adding nodes with edges to previously added nodes (see \hyperref[alg:ba]{Algorithm 3}). 
    Both models require sampling of an integer in an interval for each of its edges (or more).
    \item An Alias-Table~\cite{Alias} can be used to simulate static discrete distributions in practice. 
    After creating and initializing the table, we can sample in (expected) time $\Theta(1)$ by drawing a uniform random row and offset in each round. 
    Hence, we need to sample an integer uniformly at random in $[0,n)$ where $n$ is the number of rows in the table.
\end{itemize}


%%% BEGIN Fisher-Yates-Shuffle
\begin{algorithm}[!htb] \label{alg:fy}
    \caption{Fisher-Yates-Shuffle: uniformly shuffle an array of size $n$}
    \Require{source of uniformly-distrbuted random integers in $[0,i]$ for parameter $i$}
    \Require{array $A$ of size $n$}
    \For{$i = n - 1,\ldots,1$}{
        $j \leftarrow$ random integer in $[0,i]$\;
        swap $A[i]$ and $A[j]$\;
    }
\end{algorithm}
%%% END Fisher-Yates-Shuffle


%%% BEGIN G(n,m) Sampling
\begin{algorithm}[!htb] \label{alg:gnm}
    \caption{Sampling of (directed) $\mathcal{G}(n,m)$ graphs using Reservoir Sampling}
    \Require{source of uniformly-distrbuted random integers in $[0,i]$ for parameter $i$}
    \Require{number of nodes $n$ and number of edges $m \leq n^2$}
    $E \leftarrow$ array (of edges) of size $m$\;
    \For{$i = 0,\ldots,m - 1$}{
        $E[i] \leftarrow \left(i \div n, i \mod n\right)$\;
    }
    \For{$i = m,\ldots,n^2$}{
        $j \leftarrow$ random integer in $[0,i]$\;
        \If{$j < m$}{
            $E[j] \leftarrow \left(i \div n, i \mod n\right)$\;
        }
    }
    \KwRet{$E$}\;
\end{algorithm}
%%% END G(n,m) Sampling

%%% BEGIN BA-Sampling
\begin{algorithm}[!htb] \label{alg:ba}
    \caption{Preferential-Attachment-Generator \cite{BASampling}: sample BA-graphs in linear time}
    \Require{source of uniformly-distrbuted random integers in $[0,i]$ for parameter $i$}
    \Require{inital edge-list $E_0$ with $n_0$ nodes and $m_0 \coloneqq |E_0|$ edges, number of additional nodes $N$, neighbor-parameter $\nu$}
    $M \leftarrow$ array of size $2\left(m_0 + \nu N\right)$\;
    \For{$i = 0,\ldots,m_0 - 1$}{
        $\left(M[2i], M[2i + 1]\right) \leftarrow E_0[i]$\;
    }
    \For{$i = 0,\ldots,N - 1$}{
        \For{$j = 1,\ldots,\nu$}{
            $k \leftarrow$ random integer in $[0,2(m_0 + i\nu) - 1]$\;
            $M[2(m_0 + i\nu)] \leftarrow M[k]$\;
            $M[2(m_0 + i\nu) + 1] \leftarrow n_0 + i$\;
        }
    }
    $E \leftarrow$ array (of edges) of size $\left(m_0 + \nu N\right)$\;
    \For{$i = 0,\ldots,m_0 + \nu N - 1$}{
        $E[i] \leftarrow (M[2i], M[2i + 1])$\;
    }
    \KwRet{$E$}\;
\end{algorithm}
%%% END BA-Sampling


\subsection{Preliminaries}
We denote by $x \div y$ the integer division operation, namely $\lfloor\nicefrac{x}{y}\rfloor$, and the remainder operation by $x \Mod y \equiv x - (x \div y)y$. 
Divisions by a power of two can be efficiently implemented by Bit-\textsc{RightShift} denoted by $\gg$: $x \div 2^W \equiv x \gg W$.
Similarly, we can efficiently compute the remainder of a division py a power of two by a logical \textsc{And}-operation, denoted by $\&$: $x \Mod 2^W \equiv x \And (2^W - 1)$.
Typically, PRNGs generate uniform random numbers in an interval $[0,2^W)$ where $W \in \{32,64\}$, namely a uniform random $W$-bit number.

Since we can map every integer in an interval $[a,b)$ to the interval $[0,b - a)$ by subtracting $a$, our goal is to generate a uniform random integer in an interval $[0,s)$ for a given $s$.
If we map all integers $x$ in $[0,2^W)$ with the function $x \Mod s$, we get the following sequence:
{\Large
\[
    \overbrace{\underbrace{\overbrace{0,1,\ldots,s - 1}^{\text{$s$ values}},\overbrace{0,1,\ldots,s - 1}^{\text{$s$ values}},\ldots,\overbrace{0,1,\ldots,s - 1}^{\text{$s$ values}}}_{\text{$\left(2^W \div s\right) \cdot s$ values}},\underbrace{0,1,\ldots,\left(2^W \Mod s\right) - 1}_{\text{$2^W \Mod s$ values}}}^{\text{$2^W$ values}}
\]
}%
This motivates the following lemma:
\begin{lemma}[Remainder Multiples]
    For given integers $a,b,s > 0$ with $a < b$, there exist exactly $(b - a) \div s$ integers $x \in [a,b)$ for every $0 \leq r < s$ such that $x \Mod s \equiv r$ whenever $s$ divides $b - a$.
\end{lemma}



\subsection{Related Work}

\subsection{Structure}
We present and analyse existing techniques for uniformly sampling random integers in an interval $[0,s)$ in \hyperref[sec:2]{Section 2}.
In \hyperref[sec:3]{Section 3}, we present the main contribution of this paper: a nearly-divisionless algorithm to sampler uniform random integers in an interval $[0,s)$.
In \hyperref[sec:4]{Section 4}, we benchmark this new technique in two experiments - shuffling and performance in the novel \textsc{Upmem-Pim} architecture - and \hyperref[sec:5]{Section 5} contains a summary of all results.


%%%%% Techniques
\section{Existing Techniques}\label{sec:2}
\subsection{Biased}

\subsection{Unbiased}


%%%%% Lemire
\section{Nearly Divisionless Unbiased Sampling}\label{sec:3}


%%%%% Experiments
\section{Experiments}\label{sec:4}
\subsection{Shuffling-Experiments}

\subsection{In-Memory-Processing-Experiments}


%%%%% Conclusion
\section{Conclusion}\label{sec:5}




\bibliography{refs}

\end{document}

\section{Preliminaries}

\subsection{Formal Definition}
\begin{frame}{Formal Definition}
    \pause
    \onslide<+->{Setting:}
    \begin{itemize}[<+->]
        \item \textbf{Input:} upper bound of interval $n \in \mathbb{N}$
        \item \textbf{Output:} uniform random integer in interval $[0,n)$
    \end{itemize}

    \vspace*{0.5cm}

    \begin{alertblock}{}<+->
        \centering
        But what if we want a random integer in $[a,b)$ for $a,b \in \mathbb{N},\,0 < a < b$ instead?
    \end{alertblock}

    \smallskip

    \onslide<+->{We can map this to our setting by subtracting $a$!}
    \begin{itemize}[<+->]
        \item Set $n = b - a$ and draw a uniform random integer $x \in [0,n)$
        \item Return $x + a$
    \end{itemize}
\end{frame}



\subsection{Operations}
\begin{frame}{Operations}
    \pause 
    \begin{definition}[Common Operations]<+->
        \begin{itemize}[<+->]
            \item \makebox[4cm][l]{Integer-Division:} \makebox[1.5cm][l]{$x \div y$} $\coloneqq \lfloor\nicefrac{x}{y}\rfloor$
            \item \makebox[4cm][l]{Remainder-Operation:} \makebox[1.5cm][l]{$x \Mod y$} $\coloneqq x - (x \div y)y$
            \item \makebox[4cm][l]{Bit-\textsc{RightShift}:} \makebox[1.5cm][l]{$x \gg W$} $\coloneqq x \div 2^W$
            \item \makebox[4cm][l]{Bit-\textsc{LeftShift}:} \makebox[1.5cm][l]{$x \ll W$} $\coloneqq x \cdot 2^W$
            \item \makebox[4cm][l]{Bitwise-\textsc{And}:} \makebox[1.5cm][l]{$x \And y$} \onslide<+->{$\rightarrow x \Mod 2^W \coloneqq x \And (2^W - 1)$} 
        \end{itemize}
    \end{definition}

\end{frame}



\subsection{The Naive Approach}
\begin{frame}{The Naive Approach}
    \pause 
    \begin{block}{How do we get random numbers?}<+->
        \begin{itemize}[<+->]
            \item Generated by Pseudo-Random-Number-Generators (PRNGs)
            \item Generated as $W$-bit words, i.e. unsigned integers in $[0,2^W)$ (typically $W \in \{32,64\}$)
        \end{itemize}
    \end{block}

    \begin{exampleblock}{}<+->
        \begin{center}
            $\texttt{rand()} \Mod n$
        \end{center}
    \end{exampleblock}

    \onslide<+->{Does this work?}
    \begin{itemize}[<+->]
        \item \GB{Yes}, the generated number is in $[0,n)$.
    \end{itemize}

    \onslide<+->{Is this efficient?}
    \begin{itemize}[<+->]
        \item \ER{No}, we require one expensive integer division operation.
    \end{itemize}

    \onslide<+->{Is the generated number uniform in $[0,n)$?}
\end{frame}

\iffalse
\begin{frame}{The Naive Approach}
    \begin{alertblock}{Is the generated number uniform in $\boldsymbol{[0,n)}$?}<+->
        \onslide<+->{Consider the case where $W = 4$ and $n = 3$.}
        \onslide<+->{Then, $[0,2^W)$ is}
        \onslide<+->{\begin{align*}
            0,1,2,3,4,5,6,7,8,9,10,11,12,13,14,15
        \end{align*}}
        \onslide<+->{Apply $\left(x \Mod 3\right)$ for every $x$ and get}
        \onslide<+->{\begin{align*}
            0,1,2,0,1,2,0,1,2,0,1,2,0,1,2,0
        \end{align*}}
        \onslide<+->{\GB{$0$} appears \ER{$6$} times, \GB{$1$} appears \ER{$5$} times, and \GB{$2$} appears \ER{$5$} times!}
    \end{alertblock}

\end{frame}
\fi

\begin{frame}{The Naive Approach}
    \pause 
    \onslide<+->{In general, applying $\left(x \Mod n\right)$ to $[0,2^W)$ yields}
    \onslide<+->{\begin{align*}
        \overbrace{\underbrace{\overbrace{0,1,\ldots,n - 1}^{\text{$n$ values}},\overbrace{0,1,\ldots,n - 1}^{\text{$n$ values}},\ldots,\overbrace{0,1,\ldots,n - 1}^{\text{$n$ values}}}_{\text{$\left(2^W \div n\right) \cdot n$ values}},\ER{\underbrace{0,1,\ldots,\left(2^W \Mod n\right) - 1}_{\text{$2^W \Mod n$ values}}}}^{\text{$2^W$ values}}
    \end{align*}}

    \onslide<+->{We have a \ER{leftover} interval that introduces bias.}

    \vspace*{0.5cm}

    \onslide<+->{\GB{Every} approach that maps every integer in $[0,2^W)$ to a single number in $[0,n)$}
    \onslide<+->{does \ER{not} generate uniform random integers in one step.}

\end{frame}
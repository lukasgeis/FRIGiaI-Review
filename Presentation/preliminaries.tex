\section{Preliminaries}

\subsection{Formal Definition}
\begin{frame}{Formal Definition}
    \pause
    \onslide<+->{Setting:}
    \begin{itemize}[<+->]
        \item \textbf{Input:} upper bound of interval $n \in \mathbb{N}$
        \item \textbf{Output:} uniform random integer in interval $[0,n)$
    \end{itemize}

    \vspace*{0.5cm}

    \begin{alertblock}{}<+->
        \centering
        But what if we want a random integer in $[a,b)$ for $a,b \in \mathbb{N},\,0 < a < b$ instead?
    \end{alertblock}

    \smallskip

    \onslide<+->{We can map this to our setting by subtracting $a$!}
    \begin{itemize}[<+->]
        \item Set $n = b - a$ and draw a uniform random integer $x \in [0,n)$
        \item Return $x + a$
    \end{itemize}
\end{frame}



\subsection{Operations}
\begin{frame}{Operations}
    \pause 
    \begin{definition}[Common Operations]<+->
        \begin{itemize}[<+->]
            \item \makebox[4cm][l]{Integer-Division:} \makebox[1.5cm][l]{$x \div y$} $\coloneqq \lfloor\nicefrac{x}{y}\rfloor$
            \item \makebox[4cm][l]{Remainder-Operation:} \makebox[1.5cm][l]{$x \Mod y$} $\coloneqq x - (x \div y)y$
            \item \makebox[4cm][l]{Bit-\textsc{RightShift}:} \makebox[1.5cm][l]{$x \gg W$} $\coloneqq x \div 2^W$
            \item \makebox[4cm][l]{Bit-\textsc{LeftShift}:} \makebox[1.5cm][l]{$x \ll W$} $\coloneqq x \cdot 2^W$
            \item \makebox[4cm][l]{Bitwise-\textsc{And}:} \makebox[1.5cm][l]{$x \And y$} \onslide<+->{$\rightarrow x \Mod 2^W \coloneqq x \And (2^W - 1)$} 
        \end{itemize}
    \end{definition}

    \vspace*{0.5cm}

    \begin{definition}[Power Remainder]<+->
        For $W,n \in \mathbb{N}$, we write $\RNW$ for $2^W \Mod n$.
    \end{definition}

\end{frame}



\subsection{The Naive Approach}\label{sec:1.3}
\begin{frame}{The Naive Approach}
    \pause 
    \begin{block}{How do we get random numbers?}<+->
        \begin{itemize}[<+->]
            \item Generated by Pseudo-Random-Number-Generators (PRNGs)
            \item Generated as $W$-bit words, i.e. unsigned integers in $[0,2^W)$ (typically $W \in \{32,64\}$)
        \end{itemize}
    \end{block}

    \onslide<+->{\begin{center}
            $\texttt{rand()} \Mod n$
    \end{center}}

    \onslide<+->{Does this work?}
    \begin{itemize}[<+->]
        \item \GB{Yes}, the generated number is in $[0,n)$.
    \end{itemize}

    \onslide<+->{Is this efficient?}
    \begin{itemize}[<+->]
        \item \ER{No}, we require one expensive integer division operation.
    \end{itemize}

    \onslide<+->{Is the generated number uniform in $[0,n)$?}
\end{frame}



\begin{frame}{The Naive Approach}
    \pause 
    \onslide<+->{In general, applying $x \Mod n$ to $[0,2^W)$ yields}
    \onslide<+->{\begin{align*}\everymath{} % Allows specific coloring
        \overbrace{\underbrace{\overbrace{0,1,\ldots,n - 1}^{\text{$n$ values}},\overbrace{0,1,\ldots,n - 1}^{\text{$n$ values}},\ldots,\overbrace{0,1,\ldots,n - 1}^{\text{$n$ values}}}_{\text{$\left(2^W \div n\right) \cdot n$ values}},\ER{\underbrace{0,1,\ldots,\RNW - 1}_{\text{$\RNW$ values}}}}^{\text{$2^W$ values}}
    \end{align*}}

    \onslide<+->{We have a \ER{leftover} interval that introduces bias.}

    \vspace*{0.5cm}

    \onslide<+->{\GB{Every} approach that maps every integer in $[0,2^W)$ to a single number in $[0,n)$}
    \onslide<+->{does \ER{not} generate uniform random integers in one step}
    \onslide<+->{whenever $n$ does not divide $2^W$.}

    \vspace*{0.5cm}

    \onslide<+->{\GB{Idea:} Use \ER{rejection sampling} to achieve uniformity!}
\end{frame}
\documentclass[aspectratio=169]{gpresentation}

\RequirePackage{marvosym}

\usepackage{mathtools}
\usepackage{nicefrac}
\newcommand{\Mod}[1]{\ \mathrm{mod}\ #1}

\newcommand{\GB}[1]{\textcolor{goetheblau}{#1}}
\newcommand{\ER}[1]{\textcolor{emorot}{#1}}
\newcommand{\DG}[1]{\textcolor{dunkelgruen}{#1}}
\newcommand{\BB}[1]{\textcolor{black}{#1}}

\newcommand{\RNW}{\ensuremath{\mathcal{R}}}

\usepackage{array}
\newcolumntype{P}[1]{>{\centering\arraybackslash}p{#1}} % Centered Multilines
\usepackage{booktabs}
\usepackage{pifont} % checkmark / cross - thanks to `https://tex.stackexchange.com/questions/42619/xmark-that-complements-the-ams-checkmark`
\newcommand{\cmark}{\ding{51}}
\newcommand{\xmark}{\ding{55}}

\usepackage{everysel}
\everymath{\color{goetheblau}}

\usepackage[mode=buildnew]{standalone}

\graphicspath{{img/}}

\tikzset{
	legend/.style = {font=\boldmath, align=left},
	every path/.style = {line width=.57mm, line cap=round},
	time/.style={very thick,|-latex, gray},
	int/.style={ultra thick,{|[scale=.5]}-{|[scale=.5]}},
}

\tikzstyle{overbrace text style}=[font=\small, above, pos=.5, yshift=3mm]
\tikzstyle{overbrace style}=[decorate,decoration={brace,raise=2mm,amplitude=3pt}]
\tikzstyle{underbrace style}=[decorate,decoration={brace,raise=2mm,amplitude=3pt,mirror},color=gray]
\tikzstyle{underbrace text style}=[font=\small, below, pos=.5, yshift=-3mm]


\usepackage{xparse}

% Thanks to `https://tex.stackexchange.com/questions/8720/overbrace-underbrace-but-with-an-arrow-instead`
\renewcommand{\arraystretch}{0.7}
\NewDocumentCommand{\overarrow}{O{0} O{\big\downarrow} m}{%
  \overset{\makebox[0pt]{\begin{tabular}{@{}c@{}}$#3$\\[0pt]\ensuremath{#2}\\[0pt]\hfil\end{tabular}}}{#1}
}


\usepackage{cancel}

\subject{Seminar Algorithms for Big Data}
\title{Fast Random Integer Generation in an Interval}
\subtitle{Based on a paper of the same title by Daniel Lemire}
\author{Lukas Geis}
\supervisor{Supervised by Dr. Manuel Penschuck}
\date{29th February 2024}
\institute{Algorithm Engineering (Prof. Dr. Ulrich Meyer)}

\begin{document}
	\begin{frame}[plain, noframenumbering]
		\titlepage
	\end{frame}

	\begin{frame}{What is the problem?}{Motivation}
    \begin{center}
        \begin{tikzpicture}[scale = 1.5]
            % One Bit
            \node<1> at (5.0,0.0) {\Large{$0$}};
            \node<2-4> at (5.0,0.0) {\Large{$1$}};
            \node<3-4> at (5.0,-0.5) {$\big\uparrow$};
            \node<3-4> at (5.0,-0.9) {uniform random bit};
            \node<4> at (4.2,-1.3) {$\uparrow$};
            \node<4> at (5.8,-1.3) {$\uparrow$};
            \node<4> at (4.2,-1.7) {$0 \sim 50\%$};
            \node<4> at (5.8,-1.7) {$1 \sim 50\%$};

            % Multiple Bits
            \node<5-> at (1.5,0.0) {\Large{$1$}};
            \node<5-> at (2.5,0.0) {\Large{$1$}};
            \node<5-> at (3.5,0.0) {\Large{$0$}};
            \node<5-> at (4.5,0.0) {\Large{$1$}};
            \node<5-> at (5.5,0.0) {\Large{$0$}};
            \node<5-> at (6.5,0.0) {\Large{$0$}};
            \node<5-> at (7.5,0.0) {\Large{$0$}};
            \node<5-> at (8.5,0.0) {\Large{$1$}};

            \node<6-7> at (5.0,-0.9) {$W = 8$ independent uniform bits};
            \node<7-> at (5.0,-2.0) {\GB{\textbf{Goal:}}};
            \node<7-> at (5.0,-2.4) {Generate a uniform integer between $100$ and $200$};

            \node<8-13> at (5.0,-0.9) {\GB{interpret} as unsigned $8$-bit integer};
            \node<14> at (5.0,-0.9) {\GB{interpret} as uniform $8$-bit integer in $[0,2^8)$};

            \node<9-> at (8.5,0.5) {$\big\downarrow$};
            \node<9-> at (8.5,0.9) {$2^0$};
            \node<10-> at (7.5,0.5) {$\big\downarrow$};
            \node<10-> at (7.5,0.9) {$2^1$};
            \node<11-> at (6.5,0.5) {$\big\downarrow$};
            \node<11-> at (6.5,0.9) {$2^2$};
            \node<12-> at (5.5,0.5) {$\big\downarrow$};
            \node<12-> at (5.5,0.9) {$2^3$};
            \node<12-> at (4.5,0.5) {$\big\downarrow$};
            \node<12-> at (4.5,0.9) {$2^4$};
            \node<12-> at (3.5,0.5) {$\big\downarrow$};
            \node<12-> at (3.5,0.9) {$2^5$};
            \node<12-> at (2.5,0.5) {$\big\downarrow$};
            \node<12-> at (2.5,0.9) {$2^6$};
            \node<12-> at (1.5,0.5) {$\big\downarrow$};
            \node<12-> at (1.5,0.9) {$2^7$};

            \node<13-> at (5.0,1.5) {$209$ in binary};

            \node[opacity=0] at (5.0,1.5) {$209$ in binary};
        \end{tikzpicture}
    \end{center}
\end{frame}

	\begin{frame}{Table of Contents}
		\tableofcontents%[pausesections]
	\end{frame}

	\section{Preliminaries}

\subsection{Formal Definition}
\begin{frame}{Formal Definition}
    \pause
    \onslide<+->{Setting:}
    \begin{itemize}[<+->]
        \item \textbf{Input:} upper bound of interval $n \in \mathbb{N}$
        \item \textbf{Output:} uniform random integer in interval $[0,n)$
    \end{itemize}

    \vspace*{0.5cm}

    \begin{alertblock}{}<+->
        \centering
        But what if we want a random integer in $[a,b)$ for $a,b \in \mathbb{N},\,0 < a < b$ instead?
    \end{alertblock}

    \smallskip

    \onslide<+->{We can map this to our setting by subtracting $a$!}
    \begin{itemize}[<+->]
        \item Set $n = b - a$ and draw a uniform random integer $x \in [0,n)$
        \item Return $x + a \in [a,b)$
    \end{itemize}
\end{frame}



\subsection{Operations}
\begin{frame}{Operations}
    \pause 
    \begin{definition}[Common Operations]<+->
        \begin{itemize}[<+->]
            \item \makebox[4cm][l]{Integer-Division:} \makebox[1.5cm][l]{$x \div y$} $\coloneqq \lfloor\nicefrac{x}{y}\rfloor$
            \item \makebox[4cm][l]{Remainder-Operation:} \makebox[1.5cm][l]{$x \Mod y$} $\coloneqq x - (x \div y)y$
            \item \makebox[4cm][l]{Bit-\textsc{RightShift}:} \makebox[1.5cm][l]{$x \gg W$} $\coloneqq x \div 2^W$
            \item \makebox[4cm][l]{Bit-\textsc{LeftShift}:} \makebox[1.5cm][l]{$x \ll W$} $\coloneqq x \cdot 2^W$
            \item \makebox[4cm][l]{Bitwise-\textsc{And}:} \makebox[1.5cm][l]{$x \And y$} \onslide<+->{$\rightarrow x \Mod 2^W \coloneqq x \And (2^W - 1)$} 
        \end{itemize}
    \end{definition}

    \vspace*{0.5cm}

    \begin{definition}[Power Remainder]<+->
        For $W,n \in \mathbb{N}$, we write $\RNW$ for $2^W \Mod n$.
    \end{definition}

\end{frame}



\subsection{The Naive Approach}\label{sec:1.3}
\begin{frame}{The Naive Approach}
    \pause 
    \begin{block}{How do we get random numbers?}<+->
        \begin{itemize}[<+->]
            \item Generated by Pseudo-Random-Number-Generators (PRNGs)
            \item Generated as $W$-bit words, i.e. unsigned integers in $[0,2^W)$ (typically $W \in \{32,64\}$)
        \end{itemize}
    \end{block}

    \onslide<+->{\begin{center}
            $\texttt{rand()} \Mod n$
    \end{center}}

    \onslide<+->{Does this work?}
    \begin{itemize}[<+->]
        \item \GB{Yes}, the generated number is in $[0,n)$.
    \end{itemize}

    \onslide<+->{Is this efficient?}
    \begin{itemize}[<+->]
        \item \ER{No}, we require one expensive integer division operation.
    \end{itemize}

    \onslide<+->{Is the generated number uniform in $[0,n)$?}
\end{frame}



\begin{frame}{The Naive Approach - Bias}
    \pause 
    \onslide<+->{In general, applying $x \Mod n$ to $[0,2^W)$ yields}
    \onslide<+->{\begin{align*}\everymath{} % Allows specific coloring
        \overbrace{\underbrace{\overbrace{0,1,\ldots,n - 1}^{\text{$n$ values}},\overbrace{0,1,\ldots,n - 1}^{\text{$n$ values}},\ldots,\overbrace{0,1,\ldots,n - 1}^{\text{$n$ values}}}_{\text{$\left(2^W \div n\right) \cdot n$ values}},\ER{\underbrace{0,1,\ldots,\RNW - 1}_{\text{$\RNW$ values}}}}^{\text{$2^W$ values}}
    \end{align*}}

    \onslide<+->{We have a \ER{leftover} interval that introduces bias.}

    \vspace*{0.5cm}

    \begin{block}{Deterministic Mappings}<+->
        \onslide<+->{\GB{Every} deterministic mapping $f\colon[0,2^W) \rightarrow [0,n)$}
        \onslide<+->{does \ER{not} generate \GB{uniform} random integers in one step}
        \onslide<+->{whenever $n$ does not divide $2^W$.}
    \end{block}
    \smallskip
    \onslide<+->{\GB{Idea:} Use \ER{rejection sampling} to achieve uniformity!}
\end{frame}

	\section{Unbiased Algorithms}

\subsection{The OpenBSD Algorithm}\label{sec:2.1}
\begin{frame}{The OpenBSD Algorithm}
    \pause 
    \begin{itemize}[<+->]
        \item We shift the \ER{rejection interval} to the left:
        \onslide<+->{\begin{align*}\everymath{} % Allows specific coloring
            \overbrace{\ER{\underbrace{0,1,\ldots,\RNW - 1}_{\text{$\RNW$ values}}},\underbrace{\overbrace{\RNW,\ldots,n - 1,0,\ldots,\RNW - 1}^{\text{$n$ values}},\ldots,\overbrace{\RNW,\ldots,n - 1,0,\ldots,\RNW - 1}^{\text{$n$ values}}}_{\text{$\left(2^W \div n\right) \cdot n$ values}}}^{\text{$2^W$ values}}
        \end{align*}}
        \item Generate a uniform random number $x \in [0,2^W)$ until $x \geq \RNW$
        \item Return $x \Mod n$
    \end{itemize}

    \begin{block}{Efficiency}<+->
        \onslide<+->{We require $2$ integer division operations:}
        \onslide<+->{one for computing $\RNW$}
        \onslide<+->{and one for computing $x \Mod n$.}
    \end{block}

\end{frame}



\subsection{The Java Algorithm}\label{sec:2.2}
\begin{frame}{The Java Algorithm}
    \pause 
    \begin{itemize}
        \item<1-> Consider $x - \left(x \Mod n\right)$ for $x \in [0,2^W)$: \only<3>{\begin{align*}\everymath{} % Allows specific coloring
            \overbrace{\overbrace{0,1,\ldots,n - 1}^{\text{$n$ values}},\overbrace{0,1,\ldots,n - 1}^{\text{$n$ values}},\ldots,\overbrace{0,1,\ldots,n - 1}^{\text{$n$ values}}}^{\text{$\left(2^W \div n\right) \cdot n$ values}},\ER{\overbrace{0,1,\ldots,\RNW - 1}^{\text{$\RNW$ values}}}
        \end{align*}} \only<4>{\begin{align*}\everymath{} % Allows specific coloring
            \overbrace{\underbrace{\overbrace{\BB{0},1,\ldots,n - 1}^{\text{$n$ values}}}_{\text{mapped to $0$}},\overbrace{0,1,\ldots,n - 1}^{\text{$n$ values}},\ldots,\overbrace{0,1,\ldots,n - 1}^{\text{$n$ values}}}^{\text{$\left(2^W \div n\right) \cdot n$ values}},\ER{\overbrace{0,1,\ldots,\RNW - 1}^{\text{$\RNW$ values}}}
        \end{align*}} \only<5>{\begin{align*}\everymath{} % Allows specific coloring
            \overbrace{\underbrace{\overbrace{\BB{0},1,\ldots,n - 1}^{\text{$n$ values}}}_{\text{mapped to $0$}},\underbrace{\overbrace{\BB{0},1,\ldots,n - 1}^{\text{$n$ values}}}_{\text{mapped to $n$}},\ldots,\overbrace{0,1,\ldots,n - 1}^{\text{$n$ values}}}^{\text{$\left(2^W \div n\right) \cdot n$ values}},\ER{\overbrace{0,1,\ldots,\RNW - 1}^{\text{$\RNW$ values}}}
        \end{align*}} \only<6>{\begin{align*}\everymath{} % Allows specific coloring
            \overbrace{\underbrace{\overbrace{\BB{0},1,\ldots,n - 1}^{\text{$n$ values}}}_{\text{mapped to $0$}},\underbrace{\overbrace{\BB{0},1,\ldots,n - 1}^{\text{$n$ values}}}_{\text{mapped to $n$}},\ldots,\underbrace{\overbrace{\BB{0},1,\ldots,n - 1}^{\text{$n$ values}}}_{\substack{\text{mapped to} \\ 2^W - n - \RNW}}}^{\text{$\left(2^W \div n\right) \cdot n$ values}},\ER{\overbrace{0,1,\ldots,\RNW - 1}^{\text{$\RNW$ values}}}
        \end{align*}} \only<7->{\begin{align*}\everymath{} % Allows specific coloring
            \overbrace{\underbrace{\overbrace{\BB{0},1,\ldots,n - 1}^{\text{$n$ values}}}_{\text{mapped to $0$}},\underbrace{\overbrace{\BB{0},1,\ldots,n - 1}^{\text{$n$ values}}}_{\text{mapped to $n$}},\ldots,\underbrace{\overbrace{\BB{0},1,\ldots,n - 1}^{\text{$n$ values}}}_{\substack{\text{mapped to} \\ 2^W - n - \RNW}}}^{\text{$\left(2^W \div n\right) \cdot n$ values}},\ER{\underbrace{\overbrace{\BB{0},1,\ldots,\RNW - 1}^{\text{$\RNW$ values}}}_{\substack{\text{mapped to} \\ 2^W - \RNW}}}
        \end{align*}}
        \item<8-> We map every number to the next-smallest multiple of $n$
        \item<9-> Only numbers in the \ER{leftover} interval get mapped to $\ER{2^W - \RNW} > 2^W - n$
        \item<10-> Algorithm: \begin{itemize}
            \item[(1)]<11-> Draw $x \in [0,2^W)$ and compute $r = x \Mod n$
            \item[(2)]<12-> Return $r$ if $x - r > 2^W - n$ else goto \textbf{\GB{(1)}}
        \end{itemize}
    \end{itemize}
\end{frame}

\begin{frame}{The Java Algorithm}
    Algorithm: \begin{itemize}
        \item[(1)] Draw $x \in [0,2^W)$ and compute $r = x \Mod n$
        \item[(2)] Return $r$ if $x - r > 2^W - n$ else goto \textbf{\GB{(1)}}
    \end{itemize}
    \pause 
    \begin{block}{Efficiency}<+->
        \begin{itemize}[<+->]
            \item At least one integer division operation
            \item Number of integer divisions operations equal to number of \GB{rounds}
            \item Return a number in \GB{round} if $x < 2^W - \RNW$
            \item Happens with probability $\frac{2^W - \RNW}{2^W} > \frac{1}{2}$
            \item Expected number of integer division operations is $\frac{2^W}{2^W - \RNW} < 2$
        \end{itemize}
    \end{block}

\end{frame}



\subsection{The Bitmask Algorithm}\label{sec:2.3}
\begin{frame}{The Bitmask Algorithm}
    \pause 
    

\end{frame}






%%% Array Drawing with Tikz instead of Equation + FastDiceRoller
\iffalse
\onslide<+->{\begin{center}
    \begin{tikzpicture}[scale = 0.7]
        \draw[draw = goetheblau] (0.0,0.0) rectangle ++(5.0,1.0);
        \draw[draw = goetheblau,overbrace style] (0.0,1.0) -- (5.0,1.0) node [overbrace text style] {\GB{$n$ values}};
        \draw[draw = goetheblau] (5.0,0.0) rectangle ++(5.0,1.0);
        \draw[draw = goetheblau,overbrace style] (5.0,1.0) -- (10.0,1.0) node [overbrace text style] {\GB{$n$ values}};
        \node at (10.5,0.5) {$\cdots$};
        \draw[draw = goetheblau] (11.0,0.0) rectangle ++(5.0,1.0);
        \draw[draw = goetheblau,overbrace style] (11.0,1.0) -- (16.0,1.0) node [overbrace text style] {\GB{$n$ values}};
        \draw[draw = emorot] (16.0,0.0) rectangle ++(3.0,1.0);
        \draw[draw = emorot,overbrace style] (16.0,1.0) -- (19.0,1.0) node [overbrace text style] {$\text{\ER{\RNW values}}$};
    \end{tikzpicture}
\end{center}}

\subsection{The Fast-Dice-Roller Algorithm}
\begin{frame}{The Fast-Dice-Roller Algorithm}
    \pause 
    \begin{itemize}[<+->]
        \item Build up $x$ bit-by-bit using uniform random bits \GB{\texttt{flip()}}
        \item Keep track of upper bound $\mathcal{B}$ for number \onslide<+->{  $\longrightarrow \quad x \in [0,\mathcal{B})$}
        \item Repeat until $\mathcal{B} \geq n$ \begin{itemize}[<+->]
            \item if $x < n$, return $x$
            \item else decrease $x$ and $\mathcal{B}$ by $n$ (\ER{rejection})
        \end{itemize}
    \end{itemize}

\end{frame}
\fi 

	\section{Lemire's Algorithm}

\subsection{Multiply-And-Shift}\label{sec:3.1}
\begin{frame}{Multiply-And-Shift}
    \pause 
    \begin{itemize}
        \item Map $\texttt{rand()}$ to $[0,n)$ divisionless with $(\texttt{rand()} \cdot n) \gg W$:
        \only<3>{\begin{align*}\large
            (\texttt{rand()} \cdot n) \gg W 
        \end{align*}} 
        \only<4>{\begin{align*}\large
            (\texttt{rand()} \cdot n) \div 2^W 
        \end{align*}}
        \only<5>{\begin{align*}\large
            (\underbrace{\texttt{rand()}}_{{} \in [0,2^W)} \cdot n) \div 2^W  
        \end{align*}}
        \only<6-8>{\begin{align*}\large
            \underbrace{(\texttt{rand()} \cdot n)}_{{} \in [0,n \cdot 2^W)} \div 2^W  
        \end{align*}}
        \only<9->{\begin{align*}\large
            \underbrace{(\texttt{rand()} \cdot n) \div 2^W}_{{} \in [0,n)}
        \end{align*}}
        \item<7-> $n < 2^W \Longrightarrow n \cdot 2^W < 2^W \cdot 2^W = 2^{2W}$
        \item<8-> $2W$ bits enough to represent $\texttt{rand()} \cdot n$  
    \end{itemize}
    \vspace*{0.5cm}
    \begin{block}{Is this uniform?}<10->
        \begin{itemize}
            \item<11-> Mapping is deterministic!
            \item<12-> Mapping can \GB{not} be uniform for all $n$!
        \end{itemize}
    \end{block}
\end{frame}

\subsection{The Algorithm}\label{sec:3.2}
\begin{frame}{The Algorithm}
    \pause 
    

\end{frame}

	\section{Summary}

\begin{frame}{Summary}
    \pause 
	\centering
	\begin{tabular}{c|P{35mm}P{35mm}P{15mm}}
		\toprule
			& expected number of integer division operations & maximum number of integer division operations & Unbiased? \\
		\midrule
		\onslide<3->{Modulo Reduction & $1$ & $1$ & \xmark} 
		\onslide<4->{\\ Multiply-and-Shift & $0$ & $0$ & \xmark}
		\onslide<5->{\\ OpenBSD & $2$ & $2$ & \cmark}
		\onslide<6->{\\ Java & $\frac{2^W}{2^W - \RNW}$ & $\infty$ & \cmark}
		\onslide<7->{\\ Bitmask & $0$ & $0$ & \cmark}
		\onslide<8->{\\ Lemire & $\frac{n}{2^W}$ & $1$ & \cmark} 
	\end{tabular}

\end{frame}


\begin{frame}[c, plain, noframenumbering]
	\renewcommand{\insertsectionnumber}{!}
	\renewcommand{\insertsection}{End of Talk}
	\sectionpage
\end{frame}
\end{document}